\begin{supertabular}{p{0.07\textwidth}p{0.06\textwidth}p{0.02\textwidth}p{0.02\textwidth}p{0.06\textwidth}}
 10-31-10\textsuperscript{} &  3WID\textsuperscript{} &  \textgreater &             , &  BRDG\textsuperscript{} \\
 01-28-11\textsuperscript{} &   n2f\textsuperscript{} &               &             , &  ROC2\textsuperscript{} \\
 06-27-15\textsuperscript{} &  MAIL\textsuperscript{} &               &             , &  KITC\textsuperscript{} \\
 12-31-15\textsuperscript{} &  JUNK\textsuperscript{} &             , &  \textgreater &  JUNK\textsuperscript{} \\
 04-16-16\textsuperscript{} &   n2f\textsuperscript{} &               &  \textgreater &  JUNK\textsuperscript{} \\
 08-16-19\textsuperscript{} &  NGTN\textsuperscript{} &               &               &                   *CS2* \\
\end{supertabular}
